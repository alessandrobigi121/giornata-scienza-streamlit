\documentclass[12pt,a4paper]{article}
\usepackage[utf8]{inputenc}
\usepackage[italian]{babel}
\usepackage{geometry}
\usepackage{setspace}
\usepackage{xcolor}
\usepackage{tcolorbox}
\usepackage{enumitem}
\usepackage{hyperref}

\geometry{margin=2.5cm}
\onehalfspacing

% Box per le indicazioni di regia (cosa fare sull'app/scena)
\newtcolorbox{regia}{
    colback=blue!5!white,
    colframe=blue!75!black,
    title=\textbf{REGIA / AZIONE},
    fontupper=\small\itshape
}

\title{\textbf{Script Completo: Dai Battimenti a Heisenberg}\\
\large Giornata della Scienza 2026}
\author{Alessandro Bigi}
\date{}

\begin{document}

\maketitle

\begin{center}
    \textbf{Durata stimata:} 20 minuti circa\\
    \textbf{Target:} Studenti (Liceo)\\
    \textbf{Supporto:} App Streamlit + 2 Diapason (opzionali ma consigliati)
\end{center}

\tableofcontents
\newpage

%%%%%%%%%%%%%%%%%%%%%%%%%%%%%%%%%%%%%%%%%%%%%%%%%%%%%%%%%%%%%%%%%%%%%%%%%%%%%%%
\section{Introduzione: Il Mistero del Suono (3 minuti)}
%%%%%%%%%%%%%%%%%%%%%%%%%%%%%%%%%%%%%%%%%%%%%%%%%%%%%%%%%%%%%%%%%%%%%%%%%%%%%%%

\begin{regia}
    Stai al centro della scena. Se hai i diapason, tienili pronti sul tavolo. 
    Se non li hai, preparati ad avviare l'app sulla sezione "Battimenti".
\end{regia}

\textbf{SPEAKER:}
Buongiorno a tutti. 

Oggi voglio raccontarvi una storia. Una storia che inizia con qualcosa di molto semplice, qualcosa che le nostre orecchie possono sentire, e finisce con uno dei concetti più profondi e spesso fraintesi di tutta la fisica moderna: il Principio di Indeterminazione di Heisenberg.

Spesso, quando sentiamo parlare di meccanica quantistica, pensiamo a qualcosa di magico, di strano, di impossibile da capire se non si è dei geni della matematica. Il Principio di Indeterminazione, in particolare, ci viene raccontato così: "Non puoi sapere dov'è una particella e quanto corre veloce allo stesso tempo". E sembra quasi un limite, un difetto della natura, o forse un difetto dei nostri strumenti di misura.

Ma oggi vi dimostrerò che non è niente di tutto questo. Vi dimostrerò che il Principio di Indeterminazione non è una magia strana delle particelle subatomiche. È una conseguenza inevitabile, matematica, di un fatto molto semplice: le cose, nell'universo, si comportano come \textit{onde}.

E per capirlo, non partiremo da elettroni invisibili. partiremo dal suono.

\begin{regia}
    Prendi il primo diapason (440 Hz).
\end{regia}

Questo è un diapason. Emette una nota purissima, un LA a 440 Hertz. Ascoltate.

\begin{regia}
    Fai suonare il diapason. Lascia un attimo di silenzio.
    Poi prendi il secondo diapason (445 Hz).
\end{regia}

Questo è un altro diapason. Sembra uguale, vero? Ma in realtà è leggermente diverso. Emette un suono a 445 Hertz. La differenza è piccolissima ai nostri occhi, e anche alle nostre orecchie se li sentiamo separati.

Ma cosa succede se li facciamo suonare insieme? Ascoltate bene.

\begin{regia}
    Fai suonare entrambi i diapason contemporaneamente e avvicinali al microfono o tienili in alto.
    (Se non hai i diapason fisici, dì: "Immaginiamo di farli suonare insieme, o meglio, usiamo la nostra applicazione per simularlo fedelmente.")
\end{regia}

Lo sentite quel "Wah-wah-wah"? Il suono non è fermo. Pulsa. Va su e giù di intensità. 
Questo fenomeno si chiama \textbf{battimento}. 

Ed è proprio da questo "Wah-wah" che inizia il nostro viaggio verso la meccanica quantistica.

%%%%%%%%%%%%%%%%%%%%%%%%%%%%%%%%%%%%%%%%%%%%%%%%%%%%%%%%%%%%%%%%%%%%%%%%%%%%%%%
\section{Parte 1: Ascoltare la Matematica (5 minuti)}
%%%%%%%%%%%%%%%%%%%%%%%%%%%%%%%%%%%%%%%%%%%%%%%%%%%%%%%%%%%%%%%%%%%%%%%%%%%%%%%

\begin{regia}
    Vai al computer. Apri l'App Streamlit.
    Seleziona la sezione \textbf{"Battimenti"} dalla barra laterale.
    Carica il Preset: \textbf{"Diapason Standard LA 440 Hz"}.
\end{regia}

\textbf{SPEAKER:}
Vediamo cosa sta succedendo. Qui ho un'applicazione che simula esattamente quello che abbiamo appena sentito.

Guardate lo schermo. 
Nel primo grafico in alto vedete l'onda del primo diapason: è un'onda sinusoidale perfetta, blu. Oscilla 440 volte al secondo.
Sotto, in rosso, c'è il secondo diapason. Oscilla 445 volte al secondo. Sembrano identiche, ma la rossa è leggermente più "veloce".

Ora, la fisica ci dice una cosa fondamentale: il Principio di Sovrapposizione. Quando due suoni si incontrano nell'aria, si sommano. Semplicemente. Se uno spinge l'aria in avanti e l'altro la spinge indietro, si cancellano. Se entrambi spingono, si rinforzano.

Guardate il grafico in basso, quello viola. Questa è la somma. 

\begin{regia}
    Indica col mouse il grafico della "Somma". Mostra come l'ampiezza cresce e decresce.
    Premi il pulsante \textbf{"Genera battimenti audio"} per far sentire il suono dall'app.
\end{regia}

Vedete quella forma? Il suono non ha più la stessa ampiezza costante. Cresce, raggiunge un massimo, poi diminuisce fino quasi a sparire, e poi ricresce. Quella linea arancione tratteggiata che vedete si chiama \textbf{inviluppo}. È la forma del "Wah-wah".

E la cosa incredibile è che possiamo prevedere esattamente quanto veloce sarà questo battimento. È pura matematica: la frequenza del battimento è la \textit{differenza} tra le due frequenze.
445 meno 440 fa 5.
Quindi sentiamo 5 pulsazioni al secondo.

\begin{regia}
    Modifica nell'app la frequenza f2 portandola a \textbf{441 Hz}.
\end{regia}

Se avvicino le frequenze, diciamo 440 e 441... la differenza è 1. Sentiamo una sola pulsazione al secondo. Il battimento rallenta.

\begin{regia}
    Fai suonare l'audio o osserva il grafico che si "allarga".
\end{regia}

Cosa abbiamo imparato fin qui? Una cosa semplice ma potente: \textbf{Sommando onde diverse, creiamo una struttura}. 
Le onde singole sono noiose, sono uguali da sempre e per sempre. Ma sommandole, creiamo qualcosa che cambia nel tempo, qualcosa che ha una "forma".

E qui vi faccio una domanda: se sommando 2 onde ottengo questo semplice battimento... cosa succede se ne sommo 50? O 100? Tutte con frequenze diverse?

%%%%%%%%%%%%%%%%%%%%%%%%%%%%%%%%%%%%%%%%%%%%%%%%%%%%%%%%%%%%%%%%%%%%%%%%%%%%%%%
\section{Parte 2: Costruire un Pacchetto (5 minuti)}
%%%%%%%%%%%%%%%%%%%%%%%%%%%%%%%%%%%%%%%%%%%%%%%%%%%%%%%%%%%%%%%%%%%%%%%%%%%%%%%

\begin{regia}
    Passa alla sezione \textbf{"Pacchetti d'Onda"} nell'App.
    Seleziona il Preset: \textbf{"Pacchetto Standard"}.
\end{regia}

\textbf{SPEAKER:}
Proviamo. Immaginiamo di avere un coro di 50 diapason, ognuno accordato su una frequenza leggermente diversa, uno dopo l'altro: 100 Hz, 101 Hz, 102 Hz... fino a 130 Hz.

Che suono farebbero tutti insieme? Sarebbe solo un gran rumore?
Guardiamo il risultato.

\begin{regia}
    Mostra il grafico del "Pacchetto d'Onda risultante".
\end{regia}

Incredibile, vero? Non è un rumore caotico. Tutte quelle onde, sommate insieme, si sono cancellate quasi ovunque... tranne che in un punto centrale.
Hanno creato quello che in fisica chiamiamo un \textbf{Pacchetto d'Onda}.

Guardatelo: è un "impulso". È un'onda che esiste solo per un breve istante, e poi sparisce. È localizzata.
Qui c'è tanta energia, e fuori non ce n'è.

\begin{regia}
    Clicca su \textbf{"Mostra onde componenti singole"} per far vedere il groviglio di linee colorate sotto.
\end{regia}

Vedete sotto? Ognuna di quelle linee colorate è un'onda singola che va avanti all'infinito. Ma siccome hanno frequenze diverse, in certi punti vanno a tempo (e si sommano), in altri vanno in controtempo (e si cancellano).
Il risultato è che l'onda sopravvive solo in quel piccolo pacchetto al centro.

Tenete a mente questo passaggio perché è cruciale: \textbf{Per costruire un oggetto localizzato (che sta in un posto preciso), ho bisogno di sommare tante frequenze diverse.}

%%%%%%%%%%%%%%%%%%%%%%%%%%%%%%%%%%%%%%%%%%%%%%%%%%%%%%%%%%%%%%%%%%%%%%%%%%%%%%%
\section{Parte 3: La Coperta Corta (5 minuti)}
%%%%%%%%%%%%%%%%%%%%%%%%%%%%%%%%%%%%%%%%%%%%%%%%%%%%%%%%%%%%%%%%%%%%%%%%%%%%%%%

\textbf{SPEAKER:}
Ora arriviamo al cuore del problema. 
Vogliamo fare un esperimento con la nostra app. Vogliamo creare un pacchetto d'onda "perfetto": un pacchetto piccolissimo, strettissimo, che duri un istante brevissimo. Vogliamo sapere \textit{esattamente} quando avviene il suono.

\begin{regia}
    Seleziona il Preset \textbf{"Super-Localizzato"}.
    Indica il grafico temporale (l'onda è molto stretta).
    Poi indica il grafico dello spettro/frequenze (la banda blu è molto larga).
\end{regia}

Ecco fatto. Abbiamo un pacchetto strettissimo (la "delta x" è piccola). Ma guardate cosa è successo alle frequenze. Per ottenerlo, ho dovuto usare tantissime frequenze diverse, da 100 a 200 Hz! La mia "banda" si è allargata enormemente.

Proviamo il contrario. Diciamo: "Non mi importa della precisione nel tempo, voglio un suono puro, una nota precisissima".

\begin{regia}
    Seleziona il Preset \textbf{"Quasi-Monocromatico"}.
    Indica il grafico: l'onda ora è lunghissima, non finisce più.
    Ma la banda di frequenze è strettissima.
\end{regia}

Vedete? Ora ho poche frequenze vicine. Il "colore" del suono è puro. Ma il pacchetto si è "spalmato" ovunque. È diventato larghissimo. Non so più dire \textit{quando} c'è il suono, perché il suono è un po' ovunque.

Questo, ragazzi, non è un difetto del computer. È una legge matematica ferrea. È come una coperta troppo corta: se la tiri per coprirti la testa (precisione nella frequenza), ti scopri i piedi (precisione nel tempo). Se copri i piedi, ti scopri la testa.

Possiamo addirittura misurarlo.

\begin{regia}
    Spostati nella sezione \textbf{"Principio di Indeterminazione"}.
    Mostra i numeri in alto.
\end{regia}

Se moltiplico la larghezza del pacchetto ($\Delta x$) per la larghezza delle frequenze ($\Delta k$), ottengo sempre lo stesso numero. 
Guardate qui: circa 12.57.
Provo a stringere il pacchetto? Il numero resta 12.57.
Provo ad allargarlo? Resta 12.57.

Non c'è scampo. Non posso avere contemporaneamente $\Delta x$ zero e $\Delta k$ zero. Il loro prodotto deve essere almeno un certo valore.

%%%%%%%%%%%%%%%%%%%%%%%%%%%%%%%%%%%%%%%%%%%%%%%%%%%%%%%%%%%%%%%%%%%%%%%%%%%%%%%
\section{Parte 4: Il Salto Quantistico (4 minuti)}
%%%%%%%%%%%%%%%%%%%%%%%%%%%%%%%%%%%%%%%%%%%%%%%%%%%%%%%%%%%%%%%%%%%%%%%%%%%%%%%

\textbf{SPEAKER:}
"Va bene prof", direte voi, "interessante per gli ingegneri del suono. Ma cosa c'entra con l'Universo?"

C'entra tutto. Perché nel 1924, un giovane fisico francese di nome \textbf{Louis de Broglie} ebbe un'idea folle. Disse: "E se anche la materia... se anche gli elettroni, che noi pensiamo come palline, fossero in realtà delle onde?"

Sembrava assurdo, ma gli esperimenti gli diedero ragione. Gli elettroni, i mattoni della materia, si comportano come onde.
E de Broglie trovò la formula di traduzione tra il mondo delle particelle e il mondo delle onde.
Disse:
1. La lunghezza d'onda ($\lambda$) corrisponde alla \textbf{Quantità di Moto} ($p$), cioè alla velocità della particella.
2. Il pacchetto d'onda corrisponde alla \textbf{Posizione} della particella.

E allora... torniamo alla nostra "coperta corta".

Se l'elettrone è un'onda:
\begin{itemize}
    \item Voler sapere \textbf{dove si trova l'elettrone} significa voler fare un pacchetto \textit{stretto} nello spazio ($\Delta x$ piccolo).
    \item Ma abbiamo appena visto che per fare un pacchetto stretto, devo usare \textbf{tante frequenze} diverse ($\Delta k$ grande).
    \item E tante frequenze, per de Broglie, significano \textbf{tante velocità diverse} ($\Delta p$ grande).
\end{itemize}

Quindi: se costringo l'elettrone in un punto preciso, lui perde una velocità precisa. La sua velocità diventa incerta, sfumata.
Se invece voglio sapere esattamente la sua velocità (frequenza precisa), l'elettrone deve diventare un'onda lunghissima... e quindi non so più dove si trova! Si è spalmato in tutto lo spazio.

Eccolo qui. Il \textbf{Principio di Indeterminazione di Heisenberg}:
\[ \Delta x \cdot \Delta p \geq \frac{h}{4\pi} \]

Non è che non abbiamo microscopi abbastanza potenti. È che una particella "ferma in un punto preciso" \textit{non esiste}, proprio come non esiste un'onda "che ha una sola frequenza e dura un solo istante". È una contraddizione in termini.

%%%%%%%%%%%%%%%%%%%%%%%%%%%%%%%%%%%%%%%%%%%%%%%%%%%%%%%%%%%%%%%%%%%%%%%%%%%%%%%
\section{Conclusione (3 minuti)}
%%%%%%%%%%%%%%%%%%%%%%%%%%%%%%%%%%%%%%%%%%%%%%%%%%%%%%%%%%%%%%%%%%%%%%%%%%%%%%%

\textbf{SPEAKER:}
Per concludere.
Siamo partiti da due diapason che battevano insieme. Abbiamo visto che sommando onde creiamo pacchetti. Abbiamo scoperto che i pacchetti obbediscono a una legge di "coperta corta": o sei stretto nel tempo, o sei stretto nella frequenza.

E infine abbiamo capito che, siccome la materia è ondulatoria, anche noi, gli atomi, le stelle, siamo soggetti a questa legge. 
L'indeterminazione non è un limite alla nostra conoscenza. È la struttura stessa della realtà. È ciò che impedisce agli elettroni di collassare sul nucleo degli atomi. In un certo senso, è grazie a questa "imprecisione" fondamentale che la materia è stabile, che esiste la chimica, che esistiamo noi.

Quindi la prossima volta che sentite un'orchestra accordare gli strumenti e sentite quei "battimenti", ricordatevi: state ascoltando lo stesso meccanismo che regge la struttura fondamentale dell'universo.

Grazie.

\end{document}
