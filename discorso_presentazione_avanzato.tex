\documentclass[12pt,a4paper]{article}
\usepackage[utf8]{inputenc}
\usepackage[italian]{babel}
\usepackage{geometry}
\usepackage{setspace}
\usepackage{xcolor}
\usepackage{tcolorbox}
\usepackage{amsmath}
\usepackage{amssymb}
\usepackage{enumitem}
\usepackage{hyperref}

\geometry{margin=2.5cm}
\onehalfspacing

% Box per le indicazioni di regia (cosa fare sull'app/scena)
\newtcolorbox{regia}{
    colback=blue!5!white,
    colframe=blue!75!black,
    title=\textbf{REGIA / AZIONE},
    fonttitle=\bfseries\small,
    mapbox, % This was the error before, ensuring I don't use it or use valid options
    enhanced, % improved version if tcolorbox library 'skins' is loaded, but let's stick to simple
    fontupper=\small\itshape
}

% Redefining the box to be safe and simple like the fix in the previous step
\renewtcolorbox{regia}{
    colback=blue!5!white,
    colframe=blue!75!black,
    title=\textbf{REGIA / AZIONE},
    fonttitle=\bfseries\small,
    fontupper=\small\itshape
}

\newtcolorbox{mathnote}{
    colback=gray!5!white,
    colframe=gray!50!black,
    title=\textbf{Approfondimento Matematico},
    fonttitle=\bfseries\small,
    fontupper=\small
}

\title{\textbf{Script Avanzato: La Natura Ondulatoria della Realtà}\\
\large Dai Battimenti Acustici al Principio di Heisenberg}
\author{Alessandro Bigi}
\date{}

\begin{document}

\maketitle

\begin{center}
    \textbf{Target:} Pubblico con basi scientifiche/fisiche\\
    \textbf{Focus:} Relazioni matematiche, Trasformata di Fourier, Dualismo Onda-Particella
\end{center}

\tableofcontents
\newpage

%%%%%%%%%%%%%%%%%%%%%%%%%%%%%%%%%%%%%%%%%%%%%%%%%%%%%%%%%%%%%%%%%%%%%%%%%%%%%%%
\section{Introduzione: La Fenomenologia dell'Interferenza (3 minuti)}
%%%%%%%%%%%%%%%%%%%%%%%%%%%%%%%%%%%%%%%%%%%%%%%%%%%%%%%%%%%%%%%%%%%%%%%%%%%%%%%

\textbf{SPEAKER:}
Buongiorno. Il percorso che affronteremo oggi ha un obiettivo ambizioso: demistificare uno dei concetti più fraintesi della fisica moderna, il Principio di Indeterminazione di Heisenberg. 

Spesso viene introdotto come un assioma, o peggio, come un limite tecnologico della misura. Oggi invece vorrei mostrarvi come questo principio sia una conseguenza \textit{matematicamente inevitabile} della descrizione ondulatoria della materia. E lo faremo partendo da un fenomeno classico, macroscopico e udibile: i battimenti acustici.

\begin{regia}
    Mostra i due Diapason o avvia l'App alla sezione "Battimenti".
    Preset consigliato: \textbf{"Diapason Standard LA 440 Hz"}.
\end{regia}

Consideriamo due oscillatori armonici lineari. Due onde sonore, descrivibili come sinusoidi pure.
Abbiamo una frequenza $f_1 = 440$ Hz e una $f_2 = 445$ Hz.
Se ascoltiamo la sovrapposizione lineare di questi due segnali, percepiamo una modulazione periodica dell'intensità.

\begin{regia}
    Fai partire l'audio dei battimenti ("Genera battimenti audio").
\end{regia}

Questo è il fenomeno del battimento. Analizziamolo formalmente.

%%%%%%%%%%%%%%%%%%%%%%%%%%%%%%%%%%%%%%%%%%%%%%%%%%%%%%%%%%%%%%%%%%%%%%%%%%%%%%%
\section{Parte 1: Analisi Spettrale dei Battimenti (6 minuti)}
%%%%%%%%%%%%%%%%%%%%%%%%%%%%%%%%%%%%%%%%%%%%%%%%%%%%%%%%%%%%%%%%%%%%%%%%%%%%%%%

\textbf{SPEAKER:}
Il Principio di Sovrapposizione afferma che la funzione d'onda risultante $\psi(t)$ è la somma algebrica delle componenti.
Assumendo, per semplicità, uguale ampiezza $A$ e fase nulla:
\[ y(t) = A \cos(\omega_1 t) + A \cos(\omega_2 t) \]
dove $\omega = 2\pi f$.

Fisicamente, stiamo sommando due fasori rotanti a velocità diverse. Matematicamente, possiamo applicare le formule di prostaferesi per trasformare questa somma in un prodotto.

\begin{regia}
    \textbf{App}: Mostra il grafico dove si vede la somma (viola) e l'inviluppo (arancione).
\end{regia}

Il risultato è notevole:
\[ y(t) = \underbrace{2A \cos\left(\frac{\omega_1 - \omega_2}{2}t\right)}_{\text{Modulazione (Inviluppo)}} \cdot \underbrace{\cos\left(\frac{\omega_1 + \omega_2}{2}t\right)}_{\text{Portante}} \]

Abbiamo separato il segnale in due componenti con scale temporali diverse:
1. Una \textbf{onda portante} che oscilla alla frequenza media $\bar{\omega}$.
2. Un \textbf{inviluppo} che modula l'ampiezza a una frequenza molto più bassa, $\Delta\omega/2$.

Attenzione all'interpretazione fisica: l'orecchio (e il rivelatore quadratico) percepisce l'\textit{intensità}, che è proporzionale al quadrato dell'ampiezza.
\[ I(t) \propto y^2(t) \propto \cos^2\left(\frac{\Delta\omega}{2}t\right) \]
Poiché il coseno al quadrato ha frequenza doppia rispetto al coseno semplice, la frequenza udita del battimento è esattamente la differenza $\Delta f = |f_1 - f_2|$.

\begin{regia}
    \textbf{App}: Modifica $f_2$ portandola a 441 Hz.
\end{regia}

Riducendo il $\Delta f$ nello spazio delle frequenze, il periodo del battimento $T_{batt}$ nello spazio temporale si dilata.
\[ T_{batt} = \frac{1}{\Delta f} \]
Già qui notiamo il germe della reciprocità: $\Delta f$ piccolo implica $T$ grande. Una localizzazione precisa in frequenza (due picchi vicini) comporta una struttura temporale molto estesa.

%%%%%%%%%%%%%%%%%%%%%%%%%%%%%%%%%%%%%%%%%%%%%%%%%%%%%%%%%%%%%%%%%%%%%%%%%%%%%%%
\section{Parte 2: Sintesi di Fourier e Pacchetti d'Onda (6 minuti)}
%%%%%%%%%%%%%%%%%%%%%%%%%%%%%%%%%%%%%%%%%%%%%%%%%%%%%%%%%%%%%%%%%%%%%%%%%%%%%%%

\textbf{SPEAKER:}
Estendiamo ora il concetto. I battimenti sono la sovrapposizione di due sole frequenze discrete. Cosa accade se passiamo a uno spettro continuo?
Se sommiamo un numero infinito di onde piane con frequenze distribuite secondo una certa funzione peso $\phi(k)$, entriamo nel dominio dell'Analisi di Fourier.

\begin{regia}
    \textbf{App}: Vai a \textbf{"Pacchetti d'Onda"}.
    Seleziona Preset: \textbf{"Pacchetto Standard"}.
    Attiva: \textbf{"Mostra onde componenti"}.
\end{regia}

Costruiamo un \textbf{pacchetto d'onda}.
Matematicamente, passiamo dalla sommatoria all'integrale:
\[ \Psi(x,t) = \int_{-\infty}^{+\infty} \phi(k) e^{i(kx - \omega t)} dk \]
Dove $\phi(k)$ è la distribuzione spettrale (le ampiezze delle varie componenti $k$).

Osservate sull'App. Abbiamo una distribuzione di frequenze rettangolare (ampiezza costante tra $f_{min}$ e $f_{max}$).
La teoria delle trasformate di Fourier ci dice che la trasformata di una funzione rettangolare ("Boxcar function") è una funzione \textbf{Sinc} ($\sin(x)/x$).

Guardate la forma del pacchetto nel grafico: è effettivamente una funzione Sinc, con un lobo principale e lobi secondari che decadono.
Ciò che abbiamo ottenuto è la \textbf{localizzazione}.
Interferenza costruttiva al centro, distruttiva altrove.

%%%%%%%%%%%%%%%%%%%%%%%%%%%%%%%%%%%%%%%%%%%%%%%%%%%%%%%%%%%%%%%%%%%%%%%%%%%%%%%
\section{Parte 3: La Relazione di Reciprocità (5 minuti)}
%%%%%%%%%%%%%%%%%%%%%%%%%%%%%%%%%%%%%%%%%%%%%%%%%%%%%%%%%%%%%%%%%%%%%%%%%%%%%%%

\textbf{SPEAKER:}
Ora arriviamo al punto cruciale: la relazione tra la larghezza della distribuzione spettrale ($\Delta k$) e la larghezza spaziale del pacchetto ($\Delta x$).

\begin{regia}
    \textbf{App}: Seleziona Preset \textbf{"Super-Localizzato"}.
    Fai notare come lo spettro (banda blu) sia largo, e il pacchetto (onda) sia stretto.
\end{regia}

Per rendere il pacchetto spazialmente molto stretto ($\Delta x \to 0$), devo sommare contributi da un intervallo di frequenze ($\Delta k$) molto ampio.
Perché? Perché per "uccidere" l'onda ovunque tranne che in un punto piccolissimo, ho bisogno di un'interferenza distruttiva estremamente violenta e rapida, possibile solo se ho onde di lunghezze d'onda molto diverse che si sfasano velocemente tra loro.

\begin{regia}
    \textbf{App}: Seleziona Preset \textbf{"Quasi-Monocromatico"}.
\end{regia}

Viceversa, se restringo $\Delta k$ per avere una frequenza pura, l'interferenza diventa debole e il pacchetto si allarga all'infinito. Nel limite di un singolo $\Delta k = 0$ (onda monocromatica), $\Delta x \to \infty$. L'onda è ovunque.

Questo è un teorema matematico delle coppie di Fourier. Per qualsiasi funzione a quadrato sommabile, vale la relazione:
\[ \Delta x \cdot \Delta k \geq \text{costante} \]
Il valore della costante dipende dalla forma del pacchetto (Sinc, Gaussiano, Lorentziano), ma non può mai essere zero.
Per un pacchetto Gaussiano, che rappresenta il caso di "minima incertezza", il prodotto vale esattamente $1/2$.

%%%%%%%%%%%%%%%%%%%%%%%%%%%%%%%%%%%%%%%%%%%%%%%%%%%%%%%%%%%%%%%%%%%%%%%%%%%%%%%
\section{Parte 4: De Broglie e la Derivazione di Heisenberg (5 minuti)}
%%%%%%%%%%%%%%%%%%%%%%%%%%%%%%%%%%%%%%%%%%%%%%%%%%%%%%%%%%%%%%%%%%%%%%%%%%%%%%%

\textbf{SPEAKER:}
Tutto questo è acustica, è teoria dei segnali. Come passiamo alla Meccanica Quantistica?
Il ponte fu costruito da Louis de Broglie nel 1924.
L'ipotesi rivoluzionaria fu associare a ogni particella materiale (con quantità di moto $p$) un'onda di materia con numero d'onda $k$.

La relazione fondamentale è:
\[ p = \hbar k \quad \text{(dove } \hbar = h/2\pi \text{)} \]

Se accettiamo che la materia sia descritta da onde (funzioni d'onda $\Psi$), allora queste onde \textit{devono} obbedire alle proprietà matematiche che abbiamo appena visto sull'App.

Prendiamo la disuguaglianza matematica di Fourier:
\[ \Delta x \cdot \Delta k \geq \frac{1}{2} \]

Moltiplichiamo entrambi i membri per la costante ridotta di Planck $\hbar$:
\[ \Delta x \cdot (\hbar \Delta k) \geq \frac{\hbar}{2} \]

Poiché $\Delta p = \hbar \Delta k$ (per la linearità della derivata), otteniamo immediatamente:
\[ \Delta x \cdot \Delta p \geq \frac{\hbar}{2} \]

\begin{regia}
    \textbf{App}: Vai alla sezione \textbf{"Principio di Indeterminazione"}.
    Mostra come il prodotto $\Delta x \cdot \Delta k$ sia invariante variando i parametri.
\end{regia}

Ecco derivato il Principio di Indeterminazione di Heisenberg.
Non stiamo parlando di una perturbazione goffa causata da un fotone che colpisce un elettrone (la vecchia spiegazione "del microscopio gamma").

Stiamo dicendo una cosa molto più profonda: una particella \textbf{non possiede} simultaneamente una posizione definita e una quantità di moto definita, perché un'onda \textbf{non può} essere simultaneamente localizzata nello spazio e monocromatica in frequenza. È ontologicamente impossibile.

Se una particella "è qui" ($\Delta x$ piccolo), la sua descrizione ondulatoria richiede la sovrapposizione di infiniti stati di momento diverso. Quindi la sua quantità di moto $p$ è intrinsecamente indeterminata.

%%%%%%%%%%%%%%%%%%%%%%%%%%%%%%%%%%%%%%%%%%%%%%%%%%%%%%%%%%%%%%%%%%%%%%%%%%%%%%%
\section{Conclusione (3 minuti)}
%%%%%%%%%%%%%%%%%%%%%%%%%%%%%%%%%%%%%%%%%%%%%%%%%%%%%%%%%%%%%%%%%%%%%%%%%%%%%%%

\textbf{SPEAKER:}
Concludendo, abbiamo visto come il Principio di Indeterminazione non sia un dogma calato dall'alto, ma una proprietà strutturale delle onde.
Che si tratti di onde sonore nei battimenti, di impulsi radar, o di elettroni in un atomo, la matematica sottostante è la stessa: la Trasformata di Fourier.

L'applicazione che abbiamo usato ci permette di visualizzare questa "lotta" tra localizzazione spaziale e precisione spettrale.
La natura ondulatoria della materia ci impone questo limite fondamentale alla conoscibilità simultanea delle variabili coniugate. Non è ignoranza; è la natura stessa delle cose che, nel profondo, non sono "punti", ma "vibrazioni".

Grazie per l'attenzione.

\end{document}
