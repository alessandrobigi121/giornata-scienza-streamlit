\documentclass[12pt,a4paper]{article}
\usepackage[utf8]{inputenc}
\usepackage[italian]{babel}
\usepackage{amsmath,amssymb}
\usepackage{graphicx}
\usepackage{xcolor}
\usepackage{hyperref}
\usepackage{geometry}
\usepackage{booktabs}
\usepackage{float}
\usepackage{tcolorbox}
\usepackage{enumitem}
\usepackage{tikz}

\geometry{margin=2.5cm}

% Rimuove l'indentazione
\setlength{\parindent}{0pt}
\setlength{\parskip}{0.8em}

% Box colorati per le sezioni dell'app
\newtcolorbox{appbox}[1][]{
    colback=blue!5!white,
    colframe=blue!75!black,
    fonttitle=\bfseries,
    title={Nell'App: #1}
}

\newtcolorbox{fisicabox}[1][]{
    colback=green!5!white,
    colframe=green!50!black,
    fonttitle=\bfseries,
    title={Concetto Fisico: #1}
}

\newtcolorbox{passaggiobox}[1][]{
    colback=orange!5!white,
    colframe=orange!75!black,
    fonttitle=\bfseries,
    title={Passaggio #1}
}

\newtcolorbox{storiabox}[1][]{
    colback=purple!5!white,
    colframe=purple!50!black,
    fonttitle=\bfseries,
    title={Storia: #1}
}

\newtcolorbox{esempiobox}[1][]{
    colback=yellow!5!white,
    colframe=yellow!50!black,
    fonttitle=\bfseries,
    title={Esempio: #1}
}

\newtcolorbox{demobox}[1][]{
    colback=red!5!white,
    colframe=red!60!black,
    fonttitle=\bfseries,
    title={Demo Fisica: #1}
}

\title{\textbf{Dai Battimenti al Principio di Indeterminazione}\\
\large Guida Teorica e Pratica Completa\\
\large per la Giornata della Scienza\\
\vspace{0.5cm}
\normalsize Liceo Leopardi Majorana}
\author{Alessandro Bigi}
\date{Gennaio 2026}

\begin{document}

\maketitle

\begin{abstract}
\noindent
Questa guida accompagna la presentazione alla Giornata della Scienza (~30 minuti), illustrando il percorso concettuale che collega i \textbf{battimenti acustici} al \textbf{principio di indeterminazione di Heisenberg}. Partendo da una dimostrazione dal vivo con due diapason, arriveremo a comprendere uno dei pilastri della meccanica quantistica. L'applicazione web sviluppata permette di visualizzare ogni passaggio in modo interattivo.
\end{abstract}

\tableofcontents
\newpage

%%%%%%%%%%%%%%%%%%%%%%%%%%%%%%%%%%%%%%%%%%%%%%%%%%%%%%%%%%%%%%%%%%%%%%%%%%%%%%%
\section{Introduzione: Il Percorso Concettuale}
%%%%%%%%%%%%%%%%%%%%%%%%%%%%%%%%%%%%%%%%%%%%%%%%%%%%%%%%%%%%%%%%%%%%%%%%%%%%%%%

\subsection{Perché Questo Percorso?}

Il principio di indeterminazione di Heisenberg è spesso presentato come un mistero incomprensibile della meccanica quantistica. Gli studenti lo memorizzano come una formula ($\Delta x \cdot \Delta p \geq \hbar/2$) senza capire \textit{perché} esista questa limitazione.

L'obiettivo di questa presentazione è mostrare che il principio di indeterminazione non è affatto misterioso: è la \textbf{naturale conseguenza} del comportamento ondulatorio della materia. E questo comportamento ondulatorio può essere compreso partendo da un fenomeno acustico che tutti conoscono: i battimenti.

\subsection{Le Quattro Tappe del Viaggio}

Il percorso si articola in quattro tappe fondamentali, ciascuna costruita sulla precedente:

\begin{passaggiobox}{1 - Battimenti}
Due onde sonore con frequenze vicine si sovrappongono, creando un suono che "pulsa". Questo ci introduce al concetto fondamentale di \textbf{sovrapposizione delle onde} e mostra come dalla combinazione di onde semplici nascano strutture complesse.
\end{passaggiobox}

\begin{passaggiobox}{2 - Pacchetti d'Onda}
Se invece di due onde ne sommiamo molte (con frequenze distribuite in un intervallo), otteniamo un \textbf{pacchetto d'onda}: un'onda localizzata nello spazio e nel tempo. Questo è il modo in cui la fisica descrive le particelle.
\end{passaggiobox}

\begin{passaggiobox}{3 - Relazione Larghezza-Banda}
Scopriamo che esiste una relazione inversa inevitabile: più il pacchetto è stretto nello spazio, più è largo in frequenza (e viceversa). Questa relazione è una \textbf{proprietà matematica fondamentale} delle onde, non un limite tecnologico.
\end{passaggiobox}

\begin{passaggiobox}{4 - Principio di Indeterminazione}
Questa relazione, applicata alle particelle quantistiche (che si comportano come onde secondo de Broglie), diventa il principio di indeterminazione: non possiamo conoscere simultaneamente con precisione arbitraria sia la posizione che la quantità di moto di una particella.
\end{passaggiobox}

\subsection{Cosa Rende Unico Questo Approccio}

A differenza dell'approccio tradizionale che presenta il principio di indeterminazione come un postulato della meccanica quantistica, questo percorso:

\begin{itemize}
    \item Parte da fenomeni \textbf{udibili} e \textbf{sperimentabili}
    \item Mostra la \textbf{continuità concettuale} tra acustica e meccanica quantistica
    \item Permette di \textbf{visualizzare} ogni passaggio con l'applicazione
    \item Rende il principio di indeterminazione \textbf{inevitabile}, non misterioso
\end{itemize}

%%%%%%%%%%%%%%%%%%%%%%%%%%%%%%%%%%%%%%%%%%%%%%%%%%%%%%%%%%%%%%%%%%%%%%%%%%%%%%%
\section{Tappa 1: I Battimenti}
%%%%%%%%%%%%%%%%%%%%%%%%%%%%%%%%%%%%%%%%%%%%%%%%%%%%%%%%%%%%%%%%%%%%%%%%%%%%%%%

\subsection{Dimostrazione dal Vivo con i Diapason}

Prima di spiegare la teoria, inizia con una \textbf{dimostrazione pratica} che catturi immediatamente l'attenzione del pubblico. È il modo migliore per rendere tangibile un fenomeno fisico.

\begin{demobox}{Battimenti con Due Diapason}
\textbf{Materiali necessari}: 
\begin{itemize}
    \item Un diapason standard a 440 Hz (LA centrale)
    \item Un secondo diapason a frequenza leggermente diversa (es. 445 Hz o regolabile)
\end{itemize}

\textbf{Procedura}:
\begin{enumerate}
    \item \textbf{Fai sentire prima un solo diapason}: ``Questo è un LA a 440 Hz, il riferimento usato per accordare tutti gli strumenti.''
    \item \textbf{Fai sentire il secondo diapason da solo}: ``Questo è quasi uguale, solo 5 Hz di differenza. Sembra identico, vero?''
    \item \textbf{Ora fai suonare entrambi insieme}: ``Sentite? Il suono pulsa, va su e giù. Questi sono i \textit{battimenti}.''
    \item \textbf{Fai contare i battimenti}: ``Contate quante pulsazioni in 2 secondi... Sono circa 10, quindi 5 al secondo. Esattamente la differenza di frequenza!''
\end{enumerate}

\textbf{Messaggio chiave}: ``Due suoni quasi uguali, insieme, creano qualcosa di completamente nuovo: una pulsazione. Questo è il punto di partenza del nostro viaggio.''
\end{demobox}

Dopo questa dimostrazione, il pubblico avrà un'esperienza diretta del fenomeno. Ora puoi passare a mostrarlo nell'app e spiegare perché succede.

\subsection{Il Fenomeno nella Vita Quotidiana}

I battimenti sono un fenomeno acustico che tutti possono sperimentare. Quando due strumenti musicali suonano due note con frequenze molto vicine (ma non identiche), l'ascoltatore percepisce un suono che ``pulsa'': l'intensità aumenta e diminuisce periodicamente, creando un caratteristico effetto di ``wah-wah-wah''.

Questo effetto è usato quotidianamente dai musicisti per accordare gli strumenti. Immaginiamo un violinista che accorda la corda LA confrontandola con un diapason a 440 Hz. Se la corda è leggermente stonata (ad esempio a 442 Hz), l'ascoltatore sentirà 2 battimenti al secondo. Man mano che il violinista aggiusta la corda, i battimenti rallentano. Quando finalmente scompaiono del tutto, le due frequenze sono identiche: lo strumento è accordato.

\begin{esempiobox}{Accordatura del Pianoforte}
I tecnici che accordano i pianoforti usano i battimenti come strumento di lavoro. Per accordare perfettamente due corde, ascoltano i battimenti e regolano la tensione finché non scompaiono. Questo metodo è talmente preciso che rileva differenze di frazioni di Hz!
\end{esempiobox}


\subsection{La Fisica dei Battimenti}

\begin{fisicabox}{Sovrapposizione delle Onde}
Quando due onde si incontrano nello stesso punto dello spazio, le loro ampiezze si sommano algebricamente istante per istante. Questo è il \textbf{principio di sovrapposizione}, valido per tutte le onde lineari: sonore, luminose, oceaniche, elettromagnetiche.
\end{fisicabox}

Il principio di sovrapposizione è uno dei pilastri della fisica ondulatoria. Afferma che se $y_1(t)$ e $y_2(t)$ sono soluzioni dell'equazione delle onde, anche la loro somma $y(t) = y_1(t) + y_2(t)$ è una soluzione. Questo principio ha conseguenze profonde che esploreremo.

\subsection{Derivazione Matematica Completa}

Consideriamo due onde sonore con frequenze $f_1$ e $f_2$ (con $f_1 \approx f_2$) e stessa ampiezza $A$:

\begin{equation}
y_1(t) = A \cos(2\pi f_1 t), \quad y_2(t) = A \cos(2\pi f_2 t)
\end{equation}

Notiamo che stiamo usando la funzione coseno, che inizia al massimo per $t=0$. La scelta di coseno invece di seno è una convenzione; i risultati fisici non cambiano.

La sovrapposizione è:
\begin{equation}
y(t) = y_1(t) + y_2(t) = A \cos(2\pi f_1 t) + A \cos(2\pi f_2 t)
\end{equation}

Per semplificare questa espressione, usiamo la \textbf{formula di prostaferesi} (o formula di Werner), che afferma:
\begin{equation}
\cos\alpha + \cos\beta = 2\cos\left(\frac{\alpha + \beta}{2}\right) \cos\left(\frac{\alpha - \beta}{2}\right)
\end{equation}

Applicando questa formula con $\alpha = 2\pi f_1 t$ e $\beta = 2\pi f_2 t$:

\begin{equation}
y(t) = A \cdot 2\cos\left(\frac{2\pi f_1 t + 2\pi f_2 t}{2}\right) \cos\left(\frac{2\pi f_1 t - 2\pi f_2 t}{2}\right)
\end{equation}

Semplificando:
\begin{equation}
\boxed{y(t) = 2A \cos\left(\pi (f_1 - f_2) t\right) \cdot \cos\left(\pi (f_1 + f_2) t\right)}
\end{equation}

\subsection{Interpretazione della Formula}

Questa formula è ricca di significato fisico. Il segnale risultante è il \textbf{prodotto} di due oscillazioni:

\textbf{1. L'onda portante} (oscillazione veloce):
\begin{equation}
\cos\left(\pi (f_1 + f_2) t\right) = \cos\left(2\pi \bar{f} t\right)
\end{equation}
dove $\bar{f} = \frac{f_1 + f_2}{2}$ è la \textbf{frequenza media}. Questa è la frequenza del suono che percepiamo: il "tono" della nota.

\textbf{2. L'inviluppo} (modulazione lenta):
\begin{equation}
2A\cos\left(\pi (f_1 - f_2) t\right) = 2A\cos\left(\pi f_{batt} t\right)
\end{equation}
dove $f_{batt} = |f_1 - f_2|$ è la \textbf{frequenza di battimento}. Questa modula l'ampiezza dell'onda portante.

L'inviluppo varia tra $+2A$ e $-2A$, ma poiché l'orecchio percepisce l'intensità (proporzionale al quadrato dell'ampiezza), sentiamo \textbf{due} pulsazioni per ogni periodo dell'inviluppo. Ecco perché la frequenza con cui sentiamo i "wah" è proprio $f_{batt} = |f_1 - f_2|$ battimenti al secondo.

\subsection{Esempi Numerici Dettagliati}

\begin{esempiobox}{Diapason Accordati}
Due diapason: $f_1 = 440$ Hz (LA standard) e $f_2 = 445$ Hz.

Frequenza portante: $\bar{f} = \frac{440 + 445}{2} = 442.5$ Hz

Frequenza di battimento: $f_{batt} = |440 - 445| = 5$ Hz

Risultato: sentiamo un suono a circa 442 Hz che pulsa 5 volte al secondo.

Periodo di battimento: $T_{batt} = \frac{1}{5} = 0.2$ s = 200 ms
\end{esempiobox}

\begin{esempiobox}{Accordatura Fine}
Se $f_1 = 440$ Hz e $f_2 = 441$ Hz:

Frequenza di battimento: $f_{batt} = 1$ Hz (una pulsazione al secondo)

Questo è già molto vicino all'accordatura perfetta. Un musicista esperto può rilevare battimenti anche più lenti di 0.5 Hz!
\end{esempiobox}

\begin{esempiobox}{Stonatura Evidente}
Se $f_1 = 440$ Hz e $f_2 = 460$ Hz:

Frequenza di battimento: $f_{batt} = 20$ Hz

A questa velocità, i battimenti sono così rapidi che iniziano a fondersi in un suono "ruvido" o dissonante. Sopra circa 20-30 Hz, non percepiamo più singole pulsazioni.
\end{esempiobox}

\subsection{Caso con Ampiezze Diverse}

Cosa succede se le due onde hanno ampiezze diverse? Se $y_1 = A_1\cos(2\pi f_1 t)$ e $y_2 = A_2\cos(2\pi f_2 t)$ con $A_1 \neq A_2$:

La formula di prostaferesi non si applica direttamente, ma possiamo analizzare il risultato. L'ampiezza dell'inviluppo ora varia tra:
\begin{itemize}
    \item Massimo: $A_{max} = A_1 + A_2$ (quando le due onde sono in fase)
    \item Minimo: $A_{min} = |A_1 - A_2|$ (quando le due onde sono in opposizione di fase)
\end{itemize}

Se $A_1 \neq A_2$, l'interferenza distruttiva non è mai completa: i battimenti non scendono mai a zero. Sentiamo sempre un residuo di suono anche nei punti di minimo.

\begin{appbox}{Sezione Battimenti}
Nell'app, vai alla sezione \textbf{``Battimenti''}. 

\textbf{Esperimento 1}: Imposta $f_1 = 440$ Hz, $f_2 = 445$ Hz, $A_1 = A_2 = 1.0$. Genera l'audio: sentirai 5 battimenti al secondo con silenzi nei minimi.

\textbf{Esperimento 2}: Cambia $A_2 = 0.5$ e rigenera: ora i minimi non sono più silenziosi.

\textbf{Esperimento 3}: Avvicina le frequenze ($f_2 = 441$ Hz): i battimenti rallentano.

Osserva il grafico: la curva arancione (inviluppo) mostra la modulazione dell'ampiezza.
\end{appbox}

\subsection{Grandezze Fisiche Correlate}

L'app calcola automaticamente diverse grandezze fisiche correlate ai battimenti:

\begin{table}[H]
\centering
\begin{tabular}{lll}
\toprule
\textbf{Grandezza} & \textbf{Formula} & \textbf{Unità} \\
\midrule
Frequenza media & $\bar{f} = \frac{f_1 + f_2}{2}$ & Hz \\
Frequenza battimento & $f_{batt} = |f_1 - f_2|$ & Hz \\
Periodo onda & $T_{onda} = \frac{1}{\bar{f}}$ & s \\
Periodo battimento & $T_{batt} = \frac{1}{f_{batt}}$ & s \\
Pulsazione media & $\bar{\omega} = 2\pi\bar{f}$ & rad/s \\
Lunghezza d'onda (in aria) & $\lambda = \frac{v}{\bar{f}}$ & m \\
Numero d'onda & $k = \frac{2\pi}{\lambda}$ & rad/m \\
\bottomrule
\end{tabular}
\caption{Grandezze fisiche nei battimenti (con $v = 340$ m/s per il suono in aria)}
\end{table}

\subsection{L'Idea Chiave: Dalla Somma Nasce Struttura}

Il punto cruciale di questa sezione è il seguente: \textbf{la sovrapposizione di due onde monocromatiche (uniformi nel tempo) crea una modulazione dell'ampiezza}. Il segnale risultante non è più uniforme, ma ha una struttura temporale.

Ogni onda singola, da sola, è "infinita" e "uniforme": oscilla con la stessa ampiezza per sempre. Ma quando le combiniamo, la loro interazione crea qualcosa di nuovo: regioni di alta intensità alternate a regioni di bassa intensità.

Questa osservazione pone una domanda naturale: cosa succede se invece di 2 onde ne sommiamo 3? O 10? O 100? La risposta a questa domanda ci porterà ai pacchetti d'onda.

%%%%%%%%%%%%%%%%%%%%%%%%%%%%%%%%%%%%%%%%%%%%%%%%%%%%%%%%%%%%%%%%%%%%%%%%%%%%%%%
\section{Tappa 2: I Pacchetti d'Onda}
%%%%%%%%%%%%%%%%%%%%%%%%%%%%%%%%%%%%%%%%%%%%%%%%%%%%%%%%%%%%%%%%%%%%%%%%%%%%%%%

\subsection{L'Estensione Naturale: Da 2 a N Onde}

Nella sezione precedente abbiamo visto che la sovrapposizione di 2 onde crea una modulazione. La domanda naturale è: cosa succede se sommiamo \textbf{molte} onde con frequenze distribuite in un certo intervallo?

La risposta è sorprendente e profonda: otteniamo un \textbf{pacchetto d'onda}, cioè un'onda che è localizzata nello spazio (o nel tempo). Invece di oscillare uniformemente ovunque e per sempre, l'onda è "concentrata" in una regione limitata.

\begin{fisicabox}{Pacchetto d'Onda}
Un pacchetto d'onda è la sovrapposizione di molte onde con frequenze diverse, distribuite in un intervallo $[f_{min}, f_{max}]$. Il risultato è un'onda \textbf{localizzata}: ha un'ampiezza significativa solo in una regione limitata dello spazio o del tempo, mentre è trascurabile altrove.
\end{fisicabox}

\subsection{Costruzione Matematica}

Sommiamo $N$ onde con frequenze equidistribuite tra $f_{min}$ e $f_{max}$:

\begin{equation}
y(t) = \sum_{i=1}^{N} \frac{A}{N} \cos(2\pi f_i t)
\end{equation}

dove le frequenze sono:
\begin{equation}
f_i = f_{min} + (i-1) \cdot \frac{f_{max} - f_{min}}{N-1} = f_{min} + (i-1) \cdot \frac{\Delta f}{N-1}
\end{equation}

Il fattore $A/N$ serve a normalizzare: senza di esso, sommando più onde l'ampiezza crescerebbe senza limite.

\subsection{Limite Continuo: La Funzione Sinc}

Nel limite di molte onde ($N \to \infty$), la somma discreta diventa un integrale di Fourier:

\begin{equation}
y(t) = \int_{f_{min}}^{f_{max}} A(f) e^{i 2\pi f t} df
\end{equation}

Per uno spettro uniforme (tutte le frequenze hanno la stessa ampiezza), il risultato è proporzionale alla funzione \textbf{sinc}:

\begin{equation}
y(t) \propto \text{sinc}\left(\frac{\Delta\omega \cdot t}{2}\right) = \frac{\sin(\Delta\omega \cdot t / 2)}{\Delta\omega \cdot t / 2}
\end{equation}

dove $\Delta\omega = 2\pi \Delta f = 2\pi (f_{max} - f_{min})$ è la \textbf{larghezza di banda} in pulsazione.

\subsection{Proprietà della Funzione Sinc}

La funzione sinc (abbreviazione di "sine cardinal") ha proprietà notevoli:

\textbf{1. Picco Centrale}

Nel punto $t = 0$, tutte le onde componenti sono in fase (tutte hanno valore massimo contemporaneamente). Le ampiezze si sommano costruttivamente, creando il massimo.
\begin{equation}
\text{sinc}(0) = \lim_{x \to 0} \frac{\sin x}{x} = 1
\end{equation}

\textbf{2. Zeri Regolari}

La funzione sinc si annulla quando $\sin(\Delta\omega t / 2) = 0$, cioè quando:
\begin{equation}
\frac{\Delta\omega \cdot t}{2} = n\pi, \quad n = \pm 1, \pm 2, \pm 3, \ldots
\end{equation}

I primi zeri sono a $t = \pm \frac{2\pi}{\Delta\omega}$.

\textbf{3. Lobi Laterali}

Tra uno zero e l'altro ci sono "lobi" secondari con ampiezza decrescente. Il primo lobo laterale ha un'ampiezza di circa il 21.7\% del picco centrale.

\textbf{4. Larghezza del Lobo Centrale}

La distanza tra i primi due zeri (che definisce la "larghezza" del pacchetto) è:
\begin{equation}
\Delta t = \frac{4\pi}{\Delta\omega} = \frac{2}{\Delta f}
\end{equation}

\subsection{Visualizzazione: Perché il Pacchetto è Localizzato?}

Perché la sovrapposizione di tante onde crea un pacchetto localizzato? L'intuizione è la seguente:

\textbf{Al centro} ($t = 0$): Tutte le onde sono in fase. I loro picchi si allineano e si sommano, creando un grande picco.

\textbf{Lontano dal centro} (per $|t|$ grande): Le onde hanno fasi diverse. Alcune sono positive, altre negative. In media, si cancellano a vicenda (interferenza distruttiva).

Più onde aggiungiamo, più efficace è la cancellazione lontano dal centro, e più "pulito" diventa il pacchetto.

\begin{appbox}{Sezione Pacchetti d'Onda}
Vai alla sezione \textbf{``Pacchetti d'Onda''}.

\textbf{Esperimento 1}: Seleziona "Pacchetto Standard" (50 onde, 100-130 Hz). Osserva la forma del pacchetto e i lobi laterali.

\textbf{Esperimento 2}: Attiva "Mostra onde componenti": vedi le singole onde che si sommano per creare il pacchetto.

\textbf{Esperimento 3}: Confronta presets diversi e osserva come cambia la larghezza.

Il grafico "Intensità" mostra $|y(t)|^2$, che assomiglia a una figura di diffrazione.
\end{appbox}

\subsection{Analogia con l'Ottica: Diffrazione da Fenditura}

C'è una profonda analogia tra i pacchetti d'onda e la diffrazione della luce. Quando la luce passa attraverso una fenditura rettangolare:

\begin{itemize}
    \item L'ampiezza della distribuzione delle direzioni è una funzione sinc
    \item L'intensità è $\text{sinc}^2$, con il caratteristico pattern di lobi
    \item La larghezza della fenditura e l'apertura angolare del pattern sono inversamente proporzionali
\end{itemize}

Questa non è una coincidenza: entrambi i fenomeni sono conseguenze della stessa matematica (trasformata di Fourier).

\subsection{Velocità di Gruppo e Velocità di Fase}

Per un pacchetto che si propaga nello spazio, definiamo due velocità:

\textbf{Velocità di fase}: la velocità con cui si muovono i fronti d'onda (le oscillazioni interne):
\begin{equation}
v_{fase} = \frac{\omega}{k}
\end{equation}

\textbf{Velocità di gruppo}: la velocità con cui si muove l'inviluppo (la "busta" del pacchetto):
\begin{equation}
v_{gruppo} = \frac{d\omega}{dk}
\end{equation}

Per le onde sonore in aria (mezzo non dispersivo):
\begin{equation}
v_{fase} = v_{gruppo} = v_{suono} = 340 \text{ m/s}
\end{equation}

Il pacchetto si propaga senza deformarsi. In mezzi dispersivi (es. onde su acqua profonda), le due velocità sono diverse e il pacchetto si "allarga" mentre si propaga.

%%%%%%%%%%%%%%%%%%%%%%%%%%%%%%%%%%%%%%%%%%%%%%%%%%%%%%%%%%%%%%%%%%%%%%%%%%%%%%%
\section{Tappa 3: La Relazione Fondamentale}
%%%%%%%%%%%%%%%%%%%%%%%%%%%%%%%%%%%%%%%%%%%%%%%%%%%%%%%%%%%%%%%%%%%%%%%%%%%%%%%

\subsection{L'Osservazione Cruciale}

Arrivati a questo punto, facciamo un'osservazione cruciale che è il cuore di tutta la presentazione. Confrontando diversi pacchetti d'onda, notiamo sempre lo stesso pattern:

\begin{itemize}
    \item Se la banda di frequenze è \textbf{larga} ($\Delta f$ grande), il pacchetto è \textbf{stretto} nel tempo/spazio
    \item Se la banda di frequenze è \textbf{stretta} ($\Delta f$ piccolo), il pacchetto è \textbf{largo} nel tempo/spazio
\end{itemize}

\textbf{Non esiste} un pacchetto che sia contemporaneamente stretto in frequenza E stretto nel tempo. E non è un limite tecnologico: è una proprietà matematica fondamentale.

\begin{fisicabox}{Relazione Tempo-Frequenza}
Per qualsiasi onda, esiste una relazione inversa tra la localizzazione nel tempo (o spazio) e la localizzazione in frequenza (o numero d'onda). Questa relazione è una conseguenza della trasformata di Fourier ed è valida per TUTTE le onde, indipendentemente dalla loro natura fisica.
\end{fisicabox}

\subsection{Derivazione per il Pacchetto Sinc}

Per un pacchetto sinc (spettro uniforme), la relazione è esatta. Abbiamo visto che la larghezza temporale (distanza tra i primi zeri) è:

\begin{equation}
\Delta t = \frac{4\pi}{\Delta\omega}
\end{equation}

Moltiplicando entrambi i membri per $\Delta\omega$:

\begin{equation}
\boxed{\Delta t \cdot \Delta\omega = 4\pi \approx 12.566}
\end{equation}

Questo prodotto è una \textbf{costante}! Non importa quali valori scegliamo per $\Delta t$ o $\Delta\omega$ singolarmente: il loro prodotto sarà sempre $4\pi$.

\subsection{Traduzione nello Spazio}

Per onde che si propagano nello spazio, definiamo:
\begin{itemize}
    \item $\Delta x$ = larghezza spaziale del pacchetto
    \item $\Delta k$ = larghezza nello spettro dei numeri d'onda ($k = 2\pi/\lambda$)
\end{itemize}

La relazione diventa:
\begin{equation}
\boxed{\Delta x \cdot \Delta k = 4\pi \approx 12.566}
\end{equation}

\subsection{Tabella di Verifica}

\begin{table}[H]
\centering
\begin{tabular}{lcccc}
\toprule
\textbf{Preset} & \textbf{$\Delta f$ (Hz)} & \textbf{$\Delta k$ (rad/m)} & \textbf{$\Delta x$ (m)} & \textbf{$\Delta x \cdot \Delta k$} \\
\midrule
$\Delta k$ piccolo & 5 & 0.092 & 136.1 & 12.56 \\
$\Delta k$ medio & 30 & 0.555 & 22.7 & 12.57 \\
$\Delta k$ grande & 100 & 1.848 & 6.8 & 12.57 \\
Super-localizzato & 200 & 3.70 & 3.4 & 12.57 \\
\bottomrule
\end{tabular}
\caption{Verifica della relazione $\Delta x \cdot \Delta k = 4\pi$ per diversi pacchetti}
\end{table}

Come si vede, il prodotto è sempre circa 12.57, indipendentemente dai valori individuali.

\begin{appbox}{Sezione Principio di Indeterminazione}
Vai alla sezione \textbf{``Principio di Indeterminazione''}.

\textbf{Esperimento chiave}: Prova tutti i preset e osserva:
\begin{itemize}
    \item I valori di $\Delta x$, $\Delta k$, $\Delta\omega$, $\Delta t$
    \item Il prodotto $\Delta x \cdot \Delta k$ e $\Delta\omega \cdot \Delta t$ sono sempre circa 12.57
    \item La sezione "Validazione del Metodo" mostra il confronto con la teoria
\end{itemize}

Questo dimostra sperimentalmente che la relazione è inevitabile!
\end{appbox}

\subsection{Il Limite Minimo: Diversi Tipi di Pacchetti}

Il valore $4\pi$ è specifico per i pacchetti sinc. Altri tipi di pacchetti hanno prodotti diversi. Ma esiste un limite minimo assoluto:

\begin{equation}
\Delta x \cdot \Delta k \geq \frac{1}{2}
\end{equation}

dove $\Delta x$ e $\Delta k$ sono definiti come \textbf{deviazioni standard} (RMS = Root Mean Square).

\begin{table}[H]
\centering
\begin{tabular}{lcc}
\toprule
\textbf{Tipo di Pacchetto} & \textbf{Spettro} & \textbf{$\Delta x \cdot \Delta k$} \\
\midrule
Gaussiano & Gaussiano & 0.500 (minimo!) \\
Sinc & Uniforme (rettangolare) & $4\pi \approx 12.57$ \\
Lorentziano & Lorentziano & $\approx 1$ \\
\bottomrule
\end{tabular}
\caption{Prodotto $\Delta x \cdot \Delta k$ per diversi tipi di pacchetti}
\end{table}

Il pacchetto gaussiano raggiunge il limite minimo ed è quindi il più "efficiente": per una data localizzazione spaziale, ha la minima incertezza spettrale.

\subsection{Perché Questa Relazione È Inevitabile}

La relazione inversa tra localizzazione spaziale e spettrale non è un limite tecnologico: è una \textbf{proprietà matematica} della trasformata di Fourier.

Intuitivamente: per costruire un pacchetto molto stretto, abbiamo bisogno che le onde si annullino rapidamente lontano dal centro. Ma per ottenere questa cancellazione efficiente, servono molte frequenze diverse che interferiscano distruttivamente. Quindi un pacchetto stretto implica necessariamente una banda larga.

Viceversa, se abbiamo solo poche frequenze vicine, non riusciamo a creare la cancellazione necessaria per localizzare l'onda: il pacchetto sarà inevitabilmente largo.

%%%%%%%%%%%%%%%%%%%%%%%%%%%%%%%%%%%%%%%%%%%%%%%%%%%%%%%%%%%%%%%%%%%%%%%%%%%%%%%
\section{Tappa 4: Il Principio di Indeterminazione}
%%%%%%%%%%%%%%%%%%%%%%%%%%%%%%%%%%%%%%%%%%%%%%%%%%%%%%%%%%%%%%%%%%%%%%%%%%%%%%%

\subsection{Il Salto Concettuale: Dalle Onde alla Materia}

Fin qui abbiamo parlato di onde sonore, che sono fenomeni macroscopici e intuitivi. Ma all'inizio del XX secolo, la fisica fece una scoperta rivoluzionaria: \textbf{tutte le particelle si comportano anche come onde}.

\begin{storiabox}{La Rivoluzione Quantistica}
Nel 1924, Louis de Broglie propose nella sua tesi di dottorato che non solo la luce (che si pensava fosse un'onda) si comporta a volte come particelle (fotoni), ma anche le particelle "materiali" come gli elettroni hanno proprietà ondulatorie. Questa proposta audace fu confermata sperimentalmente nel 1927 con l'esperimento di diffrazione degli elettroni di Davisson e Germer.
\end{storiabox}

\begin{fisicabox}{Dualismo Onda-Particella}
Ogni particella ha proprietà sia corpuscolari (posizione, quantità di moto) sia ondulatorie (frequenza, lunghezza d'onda). Le due descrizioni sono complementari e legate dalla \textbf{relazione di de Broglie}:
\begin{equation}
\lambda = \frac{h}{p}, \quad \text{oppure} \quad p = \frac{h}{\lambda} = \hbar k
\end{equation}
dove $h = 6.626 \times 10^{-34}$ J$\cdot$s è la costante di Planck e $\hbar = h/2\pi$.
\end{fisicabox}

\subsection{La Funzione d'Onda}

In meccanica quantistica, lo stato di una particella è descritto da una \textbf{funzione d'onda} $\psi(x)$. Il quadrato del modulo, $|\psi(x)|^2$, dà la \textbf{densità di probabilità} di trovare la particella nella posizione $x$.

Una particella localizzata (che sappiamo dove si trova) ha una funzione d'onda concentrata in una piccola regione di spazio. Ma come abbiamo visto, un'onda localizzata deve essere composta da molti numeri d'onda diversi.

E dalla relazione di de Broglie, numeri d'onda diversi significano quantità di moto diverse!

\subsection{Derivazione del Principio di Indeterminazione}

Partiamo dalla relazione che abbiamo dimostrato per le onde:
\begin{equation}
\Delta x \cdot \Delta k \geq \frac{1}{2}
\end{equation}

Dalla relazione di de Broglie, $p = \hbar k$, quindi:
\begin{equation}
\Delta p = \hbar \cdot \Delta k
\end{equation}

Sostituendo:
\begin{equation}
\Delta x \cdot \Delta p = \Delta x \cdot \hbar \Delta k = \hbar (\Delta x \cdot \Delta k) \geq \hbar \cdot \frac{1}{2}
\end{equation}

Otteniamo il \textbf{principio di indeterminazione di Heisenberg}:

\begin{equation}
\boxed{\Delta x \cdot \Delta p \geq \frac{\hbar}{2} \approx 5.27 \times 10^{-35} \text{ J$\cdot$s}}
\end{equation}

\subsection{Il Significato Fisico Profondo}

Questo principio afferma che \textbf{non è possibile} conoscere simultaneamente con precisione arbitraria sia la posizione che la quantità di moto di una particella.

È fondamentale capire che \textbf{questo non è un limite dei nostri strumenti di misura}. Non è che non siamo abbastanza bravi a misurare. È una proprietà \textbf{fondamentale} della natura, conseguenza diretta del fatto che le particelle sono descritte da onde.

Se una particella avesse simultaneamente posizione e quantità di moto ben definite, la sua funzione d'onda dovrebbe essere contemporaneamente:
\begin{itemize}
    \item Localizzata nello spazio (per avere $\Delta x$ piccolo)
    \item Localizzata nei numeri d'onda (per avere $\Delta k$ e quindi $\Delta p$ piccoli)
\end{itemize}

Ma abbiamo dimostrato che questo è matematicamente impossibile per qualsiasi onda!

\subsection{Esempi Numerici}

\begin{esempiobox}{Elettrone}
Supponiamo di localizzare un elettrone con precisione $\Delta x = 10^{-10}$ m (circa la dimensione di un atomo).

L'incertezza minima sulla quantità di moto è:
\begin{equation}
\Delta p \geq \frac{\hbar}{2\Delta x} = \frac{1.055 \times 10^{-34}}{2 \times 10^{-10}} \approx 5.3 \times 10^{-25} \text{ kg m/s}
\end{equation}

Per un elettrone ($m = 9.1 \times 10^{-31}$ kg), questo corrisponde a un'incertezza sulla velocità:
\begin{equation}
\Delta v = \frac{\Delta p}{m} \approx 5.8 \times 10^{5} \text{ m/s}
\end{equation}

Cioè circa 580 km/s! Per una particella così piccola, localizzarla anche solo a livello atomico rende la sua velocità molto incerta.
\end{esempiobox}

\begin{esempiobox}{Pallina da Tennis}
Per una pallina da tennis ($m = 0.057$ kg) localizzata con $\Delta x = 1$ mm:
\begin{equation}
\Delta p \geq \frac{1.055 \times 10^{-34}}{2 \times 10^{-3}} \approx 5.3 \times 10^{-32} \text{ kg m/s}
\end{equation}
\begin{equation}
\Delta v \approx 10^{-30} \text{ m/s}
\end{equation}

Questa incertezza è così piccola da essere completamente trascurabile. Ecco perché non notiamo effetti quantistici nella vita quotidiana!
\end{esempiobox}

\subsection{Il Collegamento Completo con i Battimenti}

Ricapitoliamo l'intero percorso concettuale:

\begin{enumerate}[label=\arabic*.]
    \item \textbf{Battimenti}: La sovrapposizione di 2 onde con frequenze vicine crea una modulazione dell'ampiezza. Dalla somma nasce struttura.
    
    \item \textbf{Pacchetti d'onda}: La sovrapposizione di N onde crea un pacchetto localizzato. Più onde sommiamo, più efficiente è la localizzazione.
    
    \item \textbf{Relazione inversa}: La localizzazione nel tempo/spazio e quella in frequenza/numero d'onda sono inversamente proporzionali. $\Delta x \cdot \Delta k \geq 1/2$ per ogni onda.
    
    \item \textbf{Dualismo onda-particella}: Le particelle quantistiche si comportano come onde, con $p = \hbar k$.
    
    \item \textbf{Principio di indeterminazione}: Sostituendo, otteniamo $\Delta x \cdot \Delta p \geq \hbar/2$.
\end{enumerate}

Il principio di indeterminazione \textbf{non} è un mistero inspiegabile: è la naturale conseguenza del comportamento ondulatorio della materia, che a sua volta abbiamo esplorato partendo da un fenomeno acustico quotidiano.

\subsection{Altre Forme del Principio di Indeterminazione}

Oltre alla coppia posizione-quantità di moto, esistono altre coppie di variabili "coniugate" che obbediscono a relazioni simili:

\begin{equation}
\Delta E \cdot \Delta t \geq \frac{\hbar}{2}
\end{equation}

Questa relazione energia-tempo è alla base di fenomeni come:
\begin{itemize}
    \item La larghezza naturale delle righe spettrali
    \item Le particelle virtuali nel vuoto quantistico
    \item L'effetto tunnel quantistico
\end{itemize}

%%%%%%%%%%%%%%%%%%%%%%%%%%%%%%%%%%%%%%%%%%%%%%%%%%%%%%%%%%%%%%%%%%%%%%%%%%%%%%%
\section{Guida Completa all'Uso dell'Applicazione}
%%%%%%%%%%%%%%%%%%%%%%%%%%%%%%%%%%%%%%%%%%%%%%%%%%%%%%%%%%%%%%%%%%%%%%%%%%%%%%%

\subsection{Struttura dell'App}

L'applicazione è organizzata in sezioni accessibili dalla sidebar:

\begin{enumerate}
    \item \textbf{Battimenti}: Visualizza e ascolta i battimenti
    \item \textbf{Pacchetti d'Onda}: Costruisci e analizza pacchetti
    \item \textbf{Spettro di Fourier}: Analizza il contenuto in frequenza
    \item \textbf{Principio di Indeterminazione}: Verifica le relazioni
    \item \textbf{Riconoscimento Battimenti}: Analizza audio dal vivo
\end{enumerate}

\subsection{Presentazione Consigliata: Scaletta Completa (~30-35 minuti)}

\subsubsection{Fase 0: Apertura con Demo Diapason (4 minuti)}

\textbf{Obiettivo}: Catturare immediatamente l'attenzione con un'esperienza sonora dal vivo.

\begin{enumerate}
    \item \textbf{Introduzione} (1 min): ``Oggi vi porterò dalle vibrazioni di un diapason fino alla meccanica quantistica. Sembra impossibile? Vedrete che no.''
    \item \textbf{Demo primo diapason} (30 sec): Fai sentire il LA a 440 Hz da solo.
    \item \textbf{Demo secondo diapason} (30 sec): Fai sentire il secondo, quasi identico.
    \item \textbf{Demo entrambi insieme} (1 min): ``Ora sentite cosa succede insieme...'' I battimenti sono evidenti.
    \item \textbf{Domanda al pubblico} (1 min): ``Quante pulsazioni contate in 2 secondi? Perché succede questo? È quello che scopriremo.''
\end{enumerate}

\subsubsection{Fase 1: Battimenti nell'App (7-8 minuti)}

\textbf{Obiettivo}: Visualizzare e spiegare ciò che hanno appena sentito.

\begin{enumerate}
    \item \textbf{Connessione} (30 sec): ``Quello che avete sentito, ora lo vediamo. Apro l'app...''
    \item \textbf{Demo audio app} (2 min): Genera battimenti con $f_1=440$, $f_2=445$ Hz. Fai ascoltare: ``Riconoscete il suono?''
    \item \textbf{Spiegazione grafico} (3 min): Mostra le due onde sinusoidali e la loro somma con l'inviluppo arancione.
    \item \textbf{La formula magica} (1 min): ``La differenza di frequenza dà i battimenti al secondo. 5 Hz di differenza = 5 pulsazioni/secondo.''
    \item \textbf{Variazioni} (1-2 min): Cambia $f_2$ a 442 Hz (rallentano), poi a 460 Hz (troppo veloci, si fondono).
\end{enumerate}

\subsubsection{Fase 2: Pacchetti d'Onda (7-8 minuti)}

\textbf{Obiettivo}: Estendere l'idea da 2 onde a molte onde, mostrando la localizzazione.

\begin{enumerate}
    \item \textbf{Domanda provocatoria} (30 sec): ``Ok, abbiamo visto 2 onde. Ma cosa succede se ne sommiamo 50? O 100?''
    \item \textbf{Demo pacchetto} (2 min): Vai alla sezione Pacchetti, mostra il ``Pacchetto Standard''. Indica il picco centrale.
    \item \textbf{Onde componenti} (2 min): Attiva ``Mostra onde componenti'': ``Vedete? Tante onde che si sommano creano un'onda localizzata.''
    \item \textbf{Confronto preset} (2-3 min): 
        \begin{itemize}
            \item ``Super-Localizzato'': ``Vedete quanto è stretto? Ma guardate quante frequenze usa.''
            \item ``Quasi-Monocromatico'': ``Ora è largo, ma usa pochissime frequenze.''
        \end{itemize}
    \item \textbf{L'osservazione chiave} (1 min): ``Avete notato? Pacchetto stretto = tante frequenze. Pacchetto largo = poche frequenze. Non si può avere entrambe le cose!''
\end{enumerate}

\subsubsection{Fase 3: Il Principio di Indeterminazione (8-10 minuti)}

\textbf{Obiettivo}: Quantificare la relazione e collegarla alla meccanica quantistica.

\begin{enumerate}
    \item \textbf{La matematica} (2 min): Mostra i valori $\Delta x$, $\Delta k$ e il loro prodotto. ``Guardate: $\Delta x \cdot \Delta k$ è sempre circa 12.57!''
    \item \textbf{Verifica} (2 min): Cambia preset in tempo reale: il prodotto resta costante.
    \item \textbf{Il salto concettuale} (3 min): ``Ora, e se vi dicessi che gli elettroni, le particelle che compongono tutto, si comportano come onde? È quello che scoprì de Broglie nel 1924.''
    \item \textbf{La formula di Heisenberg} (2-3 min): ``Se le particelle sono onde, la stessa relazione che abbiamo visto si traduce in: non possiamo conoscere contemporaneamente con precisione sia dove si trova una particella, sia quanto veloce va.''
\end{enumerate}

\subsubsection{Fase 4: Conclusioni (4-5 minuti)}

\begin{enumerate}
    \item \textbf{Ricapitolo} (2 min): ``Ricapitoliamo il viaggio: dai diapason ai battimenti, dai battimenti ai pacchetti d'onda, dalla relazione larghezza-banda a Heisenberg.''
    \item \textbf{Il messaggio} (1 min): ``Il principio di indeterminazione non è un mistero: è la conseguenza naturale del fatto che la materia si comporta come onde.''
    \item \textbf{Domande} (2+ min): Apri alle domande del pubblico.
\end{enumerate}

\textbf{Tempo totale stimato}: 30-35 minuti (flessibile in base alle domande e interazioni).


\subsection{Domande per Coinvolgere il Pubblico}

\textbf{Battimenti}:
\begin{itemize}
    \item "Avete mai sentito 'battere' due strumenti stonati?"
    \item "Come accordano i musicisti i loro strumenti?"
    \item "Cosa pensate succeda se le frequenze sono identiche?"
\end{itemize}

\textbf{Pacchetti}:
\begin{itemize}
    \item "Secondo voi, sommando più onde il pacchetto si allarga o si restringe?"
    \item "Possiamo avere un pacchetto strettissimo con solo 3 frequenze?"
\end{itemize}

\textbf{Indeterminazione}:
\begin{itemize}
    \item "Perché non notiamo questi effetti con una pallina?"
    \item "È un limite dei nostri strumenti o della natura stessa?"
\end{itemize}

\subsection{Errori Comuni da Evitare}

\begin{itemize}
    \item Non dire che l'indeterminazione è un "mistero inspiegabile"
    \item Non dire che è un limite degli strumenti di misura
    \item Non complicare con trasformata di Fourier e numeri complessi
    \item Non saltare direttamente ai pacchetti senza passare dai battimenti
\end{itemize}

%%%%%%%%%%%%%%%%%%%%%%%%%%%%%%%%%%%%%%%%%%%%%%%%%%%%%%%%%%%%%%%%%%%%%%%%%%%%%%%
\section{Schema Riassuntivo Finale}
%%%%%%%%%%%%%%%%%%%%%%%%%%%%%%%%%%%%%%%%%%%%%%%%%%%%%%%%%%%%%%%%%%%%%%%%%%%%%%%

\subsection{Mappa Concettuale}

\begin{center}
\begin{tikzpicture}[
    box/.style={rectangle, draw, rounded corners, minimum width=3cm, minimum height=1cm, text centered, font=\small, align=center},
    arrow/.style={->, thick, >=stealth}
]
    % Nodi
    \node[box, fill=blue!20] (batt) at (0,6) {BATTIMENTI\\2 onde $\to$ modulazione};
    \node[box, fill=green!20] (pkt) at (0,4) {PACCHETTI D'ONDA\\N onde $\to$ localizzazione};
    \node[box, fill=orange!20] (rel) at (0,2) {RELAZIONE INVERSA\\$\Delta x \cdot \Delta k = 4\pi$};
    \node[box, fill=red!20] (heis) at (0,0) {HEISENBERG\\$\Delta x \cdot \Delta p \geq \hbar/2$};
    
    % Frecce
    \draw[arrow] (batt) -- node[right, font=\footnotesize] {estensione} (pkt);
    \draw[arrow] (pkt) -- node[right, font=\footnotesize] {osservazione} (rel);
    \draw[arrow] (rel) -- node[right, font=\footnotesize] {de Broglie} (heis);
    
\end{tikzpicture}
\end{center}

\subsection{Formule Chiave}

\begin{table}[H]
\centering
\renewcommand{\arraystretch}{1.5}
\begin{tabular}{lll}
\toprule
\textbf{Concetto} & \textbf{Formula} & \textbf{Significato} \\
\midrule
Battimenti & $f_{batt} = |f_1 - f_2|$ & Frequenza di pulsazione \\
Pacchetto sinc & $\Delta t = \frac{4\pi}{\Delta\omega}$ & Larghezza temporale \\
Relazione tempo-freq & $\Delta t \cdot \Delta\omega = 4\pi$ & Prodotto costante \\
Relazione spazio-k & $\Delta x \cdot \Delta k = 4\pi$ & Prodotto costante \\
De Broglie & $p = \hbar k$ & Quantità di moto \\
Heisenberg & $\Delta x \cdot \Delta p \geq \frac{\hbar}{2}$ & Indeterminazione \\
\bottomrule
\end{tabular}
\caption{Formule essenziali del percorso}
\end{table}

\subsection{Schema a Punti per l'Esposizione}

\textbf{1. BATTIMENTI}
\begin{itemize}
    \item Due frequenze vicine $\to$ suono che pulsa
    \item Principio di sovrapposizione: le ampiezze si sommano
    \item Dall'uniformità nasce struttura
\end{itemize}

\textbf{2. PACCHETTI D'ONDA}
\begin{itemize}
    \item Molte frequenze insieme $\to$ onda localizzata
    \item Forma a "sinc" con picco centrale e lobi
    \item Analogia con diffrazione ottica
\end{itemize}

\textbf{3. RELAZIONE INVERSA}
\begin{itemize}
    \item Banda larga $\Leftrightarrow$ pacchetto stretto
    \item Banda stretta $\Leftrightarrow$ pacchetto largo
    \item Prodotto $\Delta x \cdot \Delta k$ costante!
    \item Non è limite tecnologico: è matematica delle onde
\end{itemize}

\textbf{4. PRINCIPIO DI INDETERMINAZIONE}
\begin{itemize}
    \item Le particelle sono onde (de Broglie, 1924)
    \item Quantità di moto $p = \hbar k$
    \item La relazione inversa $\to$ principio di Heisenberg
    \item Conseguenza inevitabile, non mistero
\end{itemize}

\subsection{Take-Home Message}

Il principio di indeterminazione di Heisenberg non è un postulato misterioso della meccanica quantistica, ma la \textbf{naturale conseguenza} del comportamento ondulatorio della materia. Le stesse leggi matematiche che governano i battimenti di due diapason governano anche il comportamento degli elettroni negli atomi.

La fisica moderna non è incomprensibile: è la naturale estensione della fisica delle onde che tutti possiamo ascoltare.

%%%%%%%%%%%%%%%%%%%%%%%%%%%%%%%%%%%%%%%%%%%%%%%%%%%%%%%%%%%%%%%%%%%%%%%%%%%%%%%
\section{Appendice: Contesto Storico}
%%%%%%%%%%%%%%%%%%%%%%%%%%%%%%%%%%%%%%%%%%%%%%%%%%%%%%%%%%%%%%%%%%%%%%%%%%%%%%%

\subsection{Cronologia}

\begin{table}[H]
\centering
\begin{tabular}{cl}
\toprule
\textbf{Anno} & \textbf{Evento} \\
\midrule
1807 & Fourier presenta la teoria delle serie \\
1900 & Planck introduce la costante $h$ \\
1905 & Einstein spiega l'effetto fotoelettrico \\
1924 & De Broglie propone il dualismo onda-particella \\
1925 & Heisenberg formula la meccanica delle matrici \\
1926 & Schrödinger formula l'equazione d'onda \\
1927 & Heisenberg pubblica il principio di indeterminazione \\
1927 & Davisson-Germer confermano la diffrazione degli elettroni \\
\bottomrule
\end{tabular}
\caption{Tappe storiche verso il principio di indeterminazione}
\end{table}

%%%%%%%%%%%%%%%%%%%%%%%%%%%%%%%%%%%%%%%%%%%%%%%%%%%%%%%%%%%%%%%%%%%%%%%%%%%%%%%
\section{Guida per gli Studenti: Usa l'App a Casa!}
%%%%%%%%%%%%%%%%%%%%%%%%%%%%%%%%%%%%%%%%%%%%%%%%%%%%%%%%%%%%%%%%%%%%%%%%%%%%%%%

Vuoi esplorare da solo i fenomeni che abbiamo visto oggi? L'applicazione è disponibile online e puoi usarla da qualsiasi computer, tablet o smartphone!

\subsection{Come Accedere all'App}

\textbf{Opzione 1: Link Diretto}

L'app è pubblicata online. Chiedi al tuo insegnante il link oppure cerca:
\begin{center}
\texttt{https://giornata-scienza.streamlit.app}
\end{center}

Non devi installare nulla: funziona direttamente nel browser (Chrome, Firefox, Safari, Edge).

\textbf{Opzione 2: QR Code}

Scansiona il QR code che trovi alla fine di questa guida con la fotocamera del tuo smartphone.

\subsection{Navigazione dell'App}

Quando apri l'app, vedrai una \textbf{sidebar} (barra laterale) sulla sinistra. Da qui puoi selezionare le diverse sezioni:

\begin{itemize}
    \item \textbf{Battimenti}: Per ascoltare e vedere i battimenti
    \item \textbf{Pacchetti d'Onda}: Per costruire pacchetti d'onda
    \item \textbf{Spettro di Fourier}: Per analizzare le frequenze
    \item \textbf{Principio di Indeterminazione}: Per verificare la relazione
    \item \textbf{Riconoscimento Battimenti}: Per analizzare audio dal vivo
\end{itemize}

Se usi uno smartphone, clicca sulla freccia in alto a sinistra per aprire la sidebar.

\subsection{Esperimenti da Fare a Casa}

Ecco alcuni esperimenti divertenti che puoi fare!

\subsubsection{Esperimento 1: Crea i Tuoi Battimenti}

\textbf{Sezione}: Battimenti

\textbf{Cosa fare}:
\begin{enumerate}
    \item Imposta $f_1 = 200$ Hz e $f_2 = 205$ Hz
    \item Clicca su "Genera Audio" e ascolta con le cuffie
    \item Conta quante pulsazioni senti in 2 secondi (dovrebbero essere 10!)
    \item Prova a cambiare $f_2$ a 201 Hz: cosa cambia?
    \item Prova con $f_2 = 220$ Hz: riesci ancora a contare i battimenti?
\end{enumerate}

\textbf{Domande da porti}:
\begin{itemize}
    \item Qual è la frequenza di battimento più lenta che riesci a percepire?
    \item Cosa succede quando i battimenti diventano troppo veloci?
\end{itemize}

\subsubsection{Esperimento 2: Costruisci un Pacchetto d'Onda}

\textbf{Sezione}: Pacchetti d'Onda

\textbf{Cosa fare}:
\begin{enumerate}
    \item Seleziona "Personalizzato" dal menu preset
    \item Imposta: $f_{min} = 100$ Hz, $f_{max} = 200$ Hz, N = 50 onde
    \item Osserva il grafico: vedi il pacchetto con il picco centrale?
    \item Ora cambia $f_{max}$ a 150 Hz (riduci la banda)
    \item Cosa succede alla larghezza del pacchetto?
    \item Prova con $f_{max} = 300$ Hz: il pacchetto è più stretto o più largo?
\end{enumerate}

\textbf{Sfida avanzata}: Riesci a trovare i valori che danno il pacchetto più stretto possibile?

\subsubsection{Esperimento 3: Verifica il Principio di Indeterminazione}

\textbf{Sezione}: Principio di Indeterminazione

\textbf{Cosa fare}:
\begin{enumerate}
    \item Prova tutti i preset disponibili
    \item Per ognuno, annota i valori di $\Delta x$, $\Delta k$, e il loro prodotto
    \item Crea una tabella come questa:
\end{enumerate}

\begin{table}[H]
\centering
\begin{tabular}{lccc}
\toprule
\textbf{Preset} & \textbf{$\Delta x$} & \textbf{$\Delta k$} & \textbf{$\Delta x \cdot \Delta k$} \\
\midrule
Super-Localizzato & ... & ... & ... \\
Standard & ... & ... & ... \\
Quasi-Monocromatico & ... & ... & ... \\
\bottomrule
\end{tabular}
\end{table}

\textbf{Domanda}: Il prodotto è sempre circa lo stesso? Perché?

\subsubsection{Esperimento 4: Analizza la Tua Voce}

\textbf{Sezione}: Spettro di Fourier

\textbf{Cosa fare}:
\begin{enumerate}
    \item Seleziona "Personalizzato" o "Onda singola"
    \item Genera il grafico dello spettro
    \item Osserva: un'onda singola ha un solo picco, un pacchetto ne ha molti!
\end{enumerate}

\textbf{Sezione}: Riconoscimento Battimenti (se disponibile)

\textbf{Cosa fare}:
\begin{enumerate}
    \item Consenti l'accesso al microfono
    \item Canta una nota (es. "Aaaaaa") vicino a un'app che suona un LA (440 Hz)
    \item L'app rileverà le due frequenze e calcolerà i battimenti!
\end{enumerate}

\subsubsection{Esperimento 5: Confronta Preset Estremi}

\textbf{Sezione}: Pacchetti d'Onda

\textbf{Cosa fare}:
\begin{enumerate}
    \item Attiva "Vista Unificata"
    \item Seleziona "Super-Localizzato": osserva il pacchetto molto stretto
    \item Poi seleziona "Quasi-Monocromatico": osserva il pacchetto molto largo
    \item Guarda anche lo spettro: quale ha più frequenze?
\end{enumerate}

\textbf{Conclusione}: È impossibile avere un pacchetto stretto con poche frequenze!

\subsection{Suggerimenti Tecnici}

\begin{itemize}
    \item \textbf{Audio}: Usa le cuffie per sentire meglio i battimenti
    \item \textbf{Grafici}: Puoi zoomare con il mouse trascinando sulla regione che ti interessa
    \item \textbf{Download}: Puoi scaricare i grafici cliccando sull'icona della fotocamera
    \item \textbf{Lentezza}: Se l'app è lenta, riduci la durata del segnale
\end{itemize}

\subsection{Sfide per i Più Curiosi}

\begin{enumerate}
    \item \textbf{Sfida musicale}: Trova due frequenze che corrispondono a due note musicali reali (es. DO e RE). Quanti battimenti producono?
    
    \item \textbf{Sfida matematica}: Calcola teoricamente $\Delta x \cdot \Delta k$ e verifica che l'app dia lo stesso risultato.
    
    \item \textbf{Sfida fisica}: Se un elettrone ha $\Delta x = 10^{-10}$ m, qual è la sua $\Delta p$ minima? Usa $\hbar = 1.055 \times 10^{-34}$ J$\cdot$s.
    
    \item \textbf{Sfida creativa}: Registra due strumenti musicali leggermente stonati e usa l'app per misurare i battimenti!
\end{enumerate}

\subsection{Glossario per lo Studente}

\begin{table}[H]
\centering
\begin{tabular}{lp{10cm}}
\toprule
\textbf{Termine} & \textbf{Significato} \\
\midrule
Frequenza ($f$) & Numero di oscillazioni al secondo. Si misura in Hertz (Hz). \\
Ampiezza ($A$) & "Altezza" dell'onda, legata all'intensità del suono. \\
Battimento & Pulsazione del suono quando due frequenze vicine si sovrappongono. \\
Pacchetto d'onda & Onda localizzata, formata dalla sovrapposizione di tante frequenze. \\
Inviluppo & Curva che "racchiude" le oscillazioni, mostra la modulazione. \\
Spettro & Grafico che mostra quali frequenze compongono un suono. \\
$\Delta x$ & Larghezza del pacchetto nello spazio. \\
$\Delta k$ & Larghezza del pacchetto nei numeri d'onda (legato alle frequenze). \\
Sinc & Funzione matematica con la forma del pacchetto d'onda. \\
\bottomrule
\end{tabular}
\caption{Glossario dei termini principali}
\end{table}

\subsection{Per Saperne di Più}

Se ti è piaciuto questo argomento, ecco alcune risorse per approfondire:

\textbf{Video consigliati}:
\begin{itemize}
    \item "Battimenti acustici" su YouTube (cerca in italiano)
    \item "Uncertainty Principle" - Veritasium (in inglese, sottotitoli disponibili)
    \item "Wave Packets" - 3Blue1Brown (in inglese, molto visuale)
\end{itemize}

\textbf{Libri divulgativi}:
\begin{itemize}
    \item "Sei pezzi facili" - Richard Feynman
    \item "L'universo elegante" - Brian Greene
    \item "Fisica quantistica per poeti" - Leon Lederman
\end{itemize}

\textbf{Per chi vuole andare oltre}:
\begin{itemize}
    \item Corsi online su Coursera o edX sulla meccanica quantistica
    \item Il libro "Fisica" di Halliday, Resnick, Walker (capitoli su onde e ottica)
\end{itemize}

\vspace{1cm}
\begin{center}
\textit{``Chi non rimane sbalordito dalla meccanica quantistica, non l'ha capita.''}\\
--- Niels Bohr
\end{center}

\end{document}
